\section{Progettazione Concettuale}
\subsection{Analisi della richiesta}
\definecolor{lightgrey}{cmyk}{0.01,0,0,0.33}

\begin{table}[h!]
    \centering
    \begin{tabular}{| p{\textwidth} |}
        \hline
        \multicolumn{1}{|c|}{\textbf{Specifiche}} \\
        \hline
        Il Corpo Nazionale dei Vigili del fuoco ha richiesto la progettazione di una base di dati relativa alla gestione operativa e amministrativa delle varie caserme dislocate nelle varie province e città italiane.
        Si vuole che ogni \textcolor{olive}{\textbf{caserma}} sia identificata da un \textit{codice numerico univoco} e siano specificati il suo \textit{indirizzo} e un \textit{numero telefonico} tramite il quale essa è raggiungibile.
        In primo luogo si rende necessario rappresentare i \textcolor{olive}{\textbf{dipendenti}} rilevanti per il contesto: \textcolor{olive}{\textbf{vigili}}, \textcolor{olive}{\textbf{centralinisti}} e \textcolor{olive}{\textbf{personale amministrativo}}.
        Per ognuno di essi, si desidera memorizzare \textit{nome}, \textit{cognome}, \textit{stipendio} percepito e \textit{caserma di riferimento}.
        Per quanto concerne i vigili, ognuno di essi possiederà uno specifico \textit{grado}; i centralinisti, oltre ai dati comuni, devono avere registrati \textit{l'anno di assunzione} e \textit{l'orario del turno lavorativo}, in quanto considerati lavoratori part-time.
        Il loro compito primario è \underline{ricevere} le \textcolor{olive}{\textbf{chiamate}} di emergenza.
        Ogni chiamata è identificata in modo univoco da un \textit{codice} e deve includere \textit{orario di inizio e fine}, il \textit{numero telefonico del chiamante} e la \textit{descrizione} sintetica dell'emergenza.
        Ogni chiamata può risolversi in due modi differenti: \underline{generare} un singolo \textcolor{olive}{\textbf{intervento}} o meno, che sarà \textit{assegnato} ad almeno una \textcolor{olive}{\textbf{squadra}}.
        Di ogni intervento si vuole tener traccia del suo \textit{tipo} e della sua \textit{priorità} nonché dell’\textit{orario di inizio} (a partire dalla fine della chiamata) e di \textit{fine}, con eventuale \textit{esito} dell’operazione.
        A supporto dell’operazione, possono essere \underline{impiegati} uno o più \textcolor{olive}{\textbf{mezzi}}, descritti per \textit{classe} (ad esempio: terrestre, aereo e marino), \textit{modello}, \textit{anno di fabbricazione} e \textit{data dell’ultima \textcolor{olive}{\textbf{manutenzione}}} effettuata.
        Queste ultime, \underline{richieste} in una certa \textit{data} dal personale amministrativo (che si caratterizza per \textit{livello funzionale}, \textit{ruolo} e può avere o meno l'\textit{abilitazione alla firma} necessaria per la richiesta di manutenzioni) e \underline{realizzate} da \textcolor{olive}{\textbf{officine}} \underline{convenzionate} in una determinata \textit{data}, riguardano uno specifico mezzo, possono essere richieste in via \textit{ordinaria} (dopo un certo tempo dall’ultima revisione) o \textit{straordinaria} (in caso di guasti) e comportano una determinata \textit{spesa}.
        Ci interessa anche avere una breve \textit{descrizione} dell’intervento.
        Ogni vigile può \underline{entrare a far parte} di una squadra (gruppo di almeno 3 vigili), la quale \underline{afferisce} ad un particolare amministrativo che la supervisiona.
        Essa inoltre è caratterizzata da un \textit{codice} che la identifica, una \textit{denominazione operativa}, \textit{l’area di competenza} e \textit{la data di costituzione del gruppo}, per determinarne l’anzianità.\\
        \hline
    \end{tabular}
\end{table}


Dopo aver letto il testo comprensivo delle richieste, procediamo con una prima strutturazione dei requisiti in
gruppi di frasi omogenee, per rendere la progettazione più semplice:
\subsection{Strutturazione dei requisiti}
\begin{table}[h!]
    \centering
    \begin{tabular}{| p{\textwidth} |}
        \hline
        \multicolumn{1}{|c|}{\textbf{Frasi di carattere generale}} \\
        \hline
        Il Corpo Nazionale dei Vigili del fuoco ha richiesto la progettazione di una base di dati relativa alla gestione operativa e amministrativa delle varie caserme dislocate nelle varie province e città italiane. \\
        \hline
    \end{tabular}
\end{table}

\begin{table}[h!]
    \centering
    \begin{tabular}{| p{\textwidth} |}
        \hline
        \multicolumn{1}{|c|}{\textbf{Frasi relative alla caserma}} \\
        \hline
        Si vuole che ogni \textcolor{olive}{\textbf{caserma}} sia identificata da un \textit{codice numerico univoco} e siano specificati il suo \textit{indirizzo} e un \textit{numero telefonico} tramite il quale essa è raggiungibile. \\
        \hline
    \end{tabular}
\end{table}

\begin{table}[h!]
    \centering
    \begin{tabular}{| p{\textwidth} |}
        \hline
        \multicolumn{1}{|c|}{\textbf{Frasi relative ai dipendenti}} \\
        \hline
        In primo luogo si rende necessario rappresentare i \textcolor{olive}{\textbf{dipendenti}} rilevanti per il contesto: \textcolor{olive}{\textbf{vigili}}, \textcolor{olive}{\textbf{centralinisti}} e \textcolor{olive}{\textbf{personale amministrativo}}. \\
        Per ognuno di essi, si desidera memorizzare \textit{nome}, \textit{cognome}, \textit{stipendio} percepito e \textit{caserma di riferimento}. \\
        \hline
    \end{tabular}
\end{table}

\begin{table}[h!]
    \centering
    \begin{tabular}{| p{\textwidth} |}
        \hline
        \multicolumn{1}{|c|}{\textbf{Frasi relative ai vigili}} \\
        \hline
        Per quanto concerne i vigili, ognuno di essi possiederà uno specifico \textit{grado}. \\
        Ogni vigile può \underline{entrare a far parte} di una squadra (gruppo di almeno 3 vigili). \\
        \hline
    \end{tabular}
\end{table}

\begin{table}[h!]
    \centering
    \begin{tabular}{| p{\textwidth} |}
        \hline
        \multicolumn{1}{|c|}{\textbf{Frasi relative ai centralinisti}} \\
        \hline
        I centralinisti, oltre ai dati comuni, devono avere registrati \textit{l'anno di assunzione} e \textit{l'orario del turno lavorativo}, in quanto considerati lavoratori part-time. \\
        Il loro compito primario è \underline{ricevere} le \textcolor{olive}{\textbf{chiamate}} di emergenza. \\
        \hline
    \end{tabular}
\end{table}

\begin{table}[h!]
    \centering
    \begin{tabular}{| p{\textwidth} |}
        \hline
        \multicolumn{1}{|c|}{\textbf{Frasi relative alle chiamate}} \\
        \hline
        Ogni chiamata è identificata in modo univoco da un \textit{codice} e deve includere \textit{orario di inizio e fine}, il \textit{numero telefonico del chiamante} e la \textit{descrizione} sintetica dell'emergenza. \\
        Ogni chiamata può risolversi in due modi differenti: \underline{generare} un singolo \textcolor{olive}{\textbf{intervento}} o meno. \\
        \hline
    \end{tabular}
\end{table}

\begin{table}[h!]
    \centering
    \begin{tabular}{| p{\textwidth} |}
        \hline
        \multicolumn{1}{|c|}{\textbf{Frasi relative agli interventi}} \\
        \hline
        Ogni chiamata può risolversi in due modi differenti: \underline{generare} un singolo \textcolor{olive}{\textbf{intervento}} o meno, che sarà \textit{assegnato} ad almeno una \textcolor{olive}{\textbf{squadra}}. \\
        Di ogni intervento si vuole tener traccia del suo \textit{tipo} e della sua \textit{priorità} nonché dell’\textit{orario di inizio} (a partire dalla fine della chiamata) e di \textit{fine}, con eventuale \textit{esito} dell’operazione. \\
        A supporto dell’operazione, possono essere \underline{impiegati} uno o più \textcolor{olive}{\textbf{mezzi}}. \\
        \hline
    \end{tabular}
\end{table}

\begin{table}[h!]
    \centering
    \begin{tabular}{| p{\textwidth} |}
        \hline
        \multicolumn{1}{|c|}{\textbf{Frasi relative ai mezzi}} \\
        \hline
        I \textcolor{olive}{\textbf{mezzi}}, descritti per \textit{classe} (ad esempio: terrestre, aereo e marino), \textit{modello}, \textit{anno di fabbricazione} e \textit{data dell’ultima \textcolor{olive}{\textbf{manutenzione}}} effettuata. \\
        \hline
    \end{tabular}
\end{table}

\begin{table}[h!]
    \centering
    \begin{tabular}{| p{\textwidth} |}
        \hline
        \multicolumn{1}{|c|}{\textbf{Frasi relative alla manutenzione e alle officine}} \\
        \hline
        Queste ultime, \underline{richieste} in una certa \textit{data} dal personale amministrativo (che si caratterizza per \textit{livello funzionale}, \textit{ruolo} e può avere o meno l'\textit{abilitazione alla firma} necessaria per la richiesta di manutenzioni) e \underline{realizzate} da \textcolor{olive}{\textbf{officine}} \underline{convenzionate} in una determinata \textit{data}. \\
        Riguardano uno specifico mezzo, possono essere richieste in via \textit{ordinaria} (dopo un certo tempo dall’ultima revisione) o \textit{straordinaria} (in caso di guasti) e comportano una determinata \textit{spesa}. \\
        Ci interessa anche avere una breve \textit{descrizione} dell’intervento. \\
        \hline
    \end{tabular}
\end{table}

\begin{table}[h!]
    \centering
    \begin{tabular}{| p{\textwidth} |}
        \hline
        \multicolumn{1}{|c|}{\textbf{Frasi relative alla squadra}} \\
        \hline
        Ogni vigile può \underline{entrare a far parte} di una squadra (gruppo di almeno 3 vigili), la quale \underline{afferisce} ad un particolare amministrativo che la supervisiona. \\
        Essa inoltre è caratterizzata da un \textit{codice} che la identifica, una \textit{denominazione operativa}, \textit{l’area di competenza} e \textit{la data di costituzione del gruppo}, per determinarne l’anzianità. \\
        \hline
    \end{tabular}
\end{table}


\subsection{Raffinazione dei concetti, costruzione dello schema concettuale}
Iniziamo con la progettazione concettuale dei vari dati, seguendo un approccio \textit{top-down}
e partendo da semplici pattern, aggiungendo mano mano quanto necessario\dots

In primo luogo siamo partiti da 3 differenti micro-schemi di questo tipo:
\begin{figure}[H]
    \centering
    \includegraphics[width=0.7\textwidth]{Schemas/Schema1.pdf}
    \caption*{Schemi 1, 2, 3}
\end{figure}

in cui abbiamo preso in considerazione un sottoinsieme delle varie entità individuando le relazioni fondamentali che le legano.
In particolare si noti come la base dovrebbe occuparsi della gestione di 3 ambiti diversi: gestione chiamate, interventi e richiesta di manutenzione.

Osserviamo come i \textbf{vigili}, i \textbf{centralinisti} e il \textbf{personale amministrativo} possano essere accorpati all’interno della generalizzazione (dalle specifiche totale e sovrapposta) \textbf{Dipendente} che riunisce le caratteristiche (attributi) comuni, come il \textit{nome}, il \textit{cognome}, lo \textit{stipendio} percepito e la \textbf{\textit{caserma}} in cui lavorano.
Risulta il seguente:
\begin{figure}[H]
    \centering
    \includegraphics[width=1.0\textwidth]{Schemas/Schema4.pdf}
    \caption*{Schema 4}
\end{figure}

A questo punto ci concentriamo sulla parte relativa a tutti gli aspetti di un intervento, notando in primo luogo come una chiamata possa generare o meno un intervento (relazione che quindi lega queste entità) e come quest’ultimo venga gestito da una squadra, ossia da un insieme di vigili che modelliamo come entità ulteriore.

Infine, la parte amministrativa. Il personale amministrativo come già detto ha due ruoli:
\begin{itemize}
    \item \textit{Richiede} le manutenzioni che verranno \textit{effettuate} da un'\textbf{officina} (che le realizza) su dei \textbf{mezzi} (modellati come entità), \textit{impiegati} nell’ambito degli interventi
    \item È responsabile della squadra, a cui essa \textit{afferisce}
\end{itemize}

A seguito di integrazione e della naturale modellazione di queste ultime tre relazioni e dell’aggiunta dell’entità \textbf{officina} che \underline{effettua} la manutenzione (nonché convenzionata con la caserma) otteniamo il seguente:

\begin{figure}[H]
    \centering
    \includegraphics[width=1.0\textwidth]{Schemas/Schema5.pdf}
    \caption*{Schema 5}
\end{figure}

Una volta terminato lo scheletro di base, aggiungiamo gli attributi alle varie entità e determiniamo le cardinalità delle partecipazioni. Lo schema risultante è il seguente:

\pagebreak
\subsection{Entità coinvolte}
Passiamo ora in rassegna quelle che sono le entità coinvolte, già \textcolor{olive}{\textbf{evidenziate}} nelle specifiche.

\begin{table}[H]
    \centering
    % Uso tabularx per adattare la tabella alla larghezza del testo
    \begin{tabularx}{\textwidth}{|l|X|X|}
        \hline
        \rowcolor{lightgrey}
        \multicolumn{1}{|c|}{\textbf{Entità}} & \multicolumn{1}{c|}{\textbf{Descrizione}} & \multicolumn{1}{c|}{\textbf{Attributi}} \\
        \hline
        \textbf{Caserma} & Struttura operativa dei vigili del fuoco dislocata sul territorio & \textbf{Codice}, Indirizzo, Numero telefonico \\
        \hline
        \textbf{Dipendente} & Generalizzazione del personale che lavora nella caserma & \textbf{ID}, Nome, Cognome, Stipendio \\
        \hline
        \textbf{Vigile} & Personale operativo che interviene nelle emergenze & Grado, Reparto \\
        \hline
        \textbf{Centralinista} & Personale che riceve le chiamate di emergenza & Anno assunzione, Orario turno \\
        \hline
        \textbf{Personale Amministrativo} & Personale che gestisce gli aspetti burocratici e amministrativi, come supervisione di interventi e ordine di revisioni su mezzi & Mansione, Livello funzionale, Abilitazione firma  \\
        \hline
        \textbf{Chiamata} & Potenziale richiesta di intervento ricevuta dalla centrale operativa & \textbf{Codice}, Orario inizio, Orario fine, Numero chiamante, Descrizione \\
        \hline
        \textbf{Intervento} & Operazione di soccorso effettuata a seguito di una chiamata valida & \textbf{Codice} Tipo, Priorità, Orario inizio, Orario fine, Esito \\
        \hline
        \textbf{Squadra} & Gruppo di vigili che opera congiuntamente negli interventi & \textbf{Codice}, Denominazione operativa, Area competenza, Data costituzione \\
        \hline
        \textbf{Mezzo} & Veicolo o attrezzatura utilizzata negli interventi & \textbf{Codice}, Classe, Modello, Anno fabbricazione, Data ultima revisione \\
        \hline
        \textbf{Manutenzione} & Attività di controllo e riparazione effettuata sui mezzi & Data richiesta, Data svolgimento, Tipo (ordinaria/straordinaria), Spesa, Descrizione \\
        \hline
        \textbf{Officina} & Struttura convenzionata che effettua le manutenzioni & \textbf{Nome, Indirizzo, Telefono} \\
        \hline
    \end{tabularx}
\end{table}

In \textbf{grassetto} abbiamo riportato gli attributi che identificano univocamente le varie entità, mentre eventuali identificazioni esterne (si veda lo schema E-R) non sono state riportate.

\subsection{Relazioni e regole di vincolo}
Di seguito riportiamo le associazioni individuate tra le entità, con le relative cardinalità derivate dall'analisi delle specifiche e i vari vincoli di integrità individuati.

\begin{table}[h!]
    \centering
    \begin{tabularx}{\textwidth}{|p{3.5cm}|p{3.5cm}|X|}
        \hline
        \rowcolor{lightgrey}
        \multicolumn{1}{|c|}{\textbf{Relazione}} & \multicolumn{1}{c|}{\textbf{Entità e Cardinalità}} & \multicolumn{1}{c|}{\textbf{Descrizione breve}} \\
        \hline
        \textbf{APPARTENENZA} & \textbf{Vigile} (1,1) \newline \textbf{Squadra} (3,N) & Un vigile appartiene ad una singola squadra, una squadra deve essere fatta da almeno 3 vigili \\
        \hline
        \textbf{ASSEGNATO A} & \textbf{Squadra} (1,1) \newline \textbf{Intervento} (1,N) & Una squadra può essere assegnata a solo un intervento, ma un intervento può essere assegnato a più squadre \\
        \hline
        \textbf{IMPIEGATO IN} & \textbf{Intervento} (1,N) \newline \textbf{Mezzo} (1,1) & In un intervento possono essere impiegati più mezzi, mentre un mezzo sarà sicuramente impiegato in un intervento \\
        \hline
        \textbf{EFFETTUATA SU} & \textbf{Manutenzione} (1,1) \newline \textbf{Mezzo} (1,1) & La manutenzione riguarda un singolo mezzo e sul mezzo viene effettuata una singola manutenzione \\
        \hline
        \textbf{REALIZZATE DA} & \textbf{Manutenzione} (1,1) \newline \textbf{Officina} (0,N) & Una manutenzione è realizzata da una sola officina, un'officina può effettuare più manutenzioni o nessuna \\
        \hline
        \textbf{RICHIEDE} & \textbf{Manutenzione} (1,1) \newline \textbf{Amministrativo} (0,N) & Un amministrativo può eventualmente (se autorizzato) richiedere una o più manutenzioni. Una manutenzione può essere richiesta da un solo amministrativo \\
        \hline
        \textbf{AFFERENZA} & \textbf{Squadra} (1,1) \newline \textbf{Amministrativo} (0,N) & Una squadra deve per forza avere un solo amministrativo che ne è responsabile. Un amministrativo può gestire anche più squadre \\
        \hline
        \textbf{RICEVE} & \textbf{Centralinista} (0,N) \newline \textbf{Chiamata} (1,1) & Ogni chiamata è ricevuta da uno e un solo centralinista, un centralinista può ricevere zero o molte chiamate \\
        \hline
        \textbf{GENERA} & \textbf{Chiamata} (0,1) \newline \textbf{Intervento} (1,1) & Una chiamata può eventualmente generare un intervento. Un intervento è necessariamente generato da una singola chiamata \\
        \hline
        \textbf{LAVORARE} & \textbf{Dipendenti} (1,1) \newline \textbf{Caserma} (1,N) & In una caserma lavora almeno 1 dipendente, un dipendente lavora in una sola caserma. \\
        \hline
        \textbf{CONVENZIONATA} & \textbf{Caserma} (1,N) \newline \textbf{Officina} (1,N) & Una caserma è convenzionata con almeno 1 officina, un officina è convenzionata con almeno una caserma \\
        \hline
    \end{tabularx}
\end{table}

\newpage

% Calcolo dinamico della larghezza: Larghezza testo - 2.5cm (per ID) - spaziatura
\begin{longtable}{|c|p{\dimexpr\textwidth-3cm\relax}|}
    \hline
    \rowcolor{lightgrey}
    \textbf{ID} & \textbf{Vincolo} \\
    \hline
    \endfirsthead
    
    \hline
    \rowcolor{lightgrey}
    \textbf{ID} & \textbf{Vincolo (continua)} \\
    \hline
    \endhead
    
    \hline
    \endfoot

    V1 & L'orario di fine di una chiamata deve essere successivo all'orario di inizio \\
    \hline
    V2 & L'orario di fine di un intervento deve essere successivo all'orario di inizio \\
    \hline
    V3 & L'orario di inizio di un intervento deve essere successivo all'orario di fine della chiamata che lo ha generato \\
    \hline
    V4 & La data di svolgimento della manutenzione deve essere successiva alla data di richiesta \\
    \hline
    V5 & L'anno di fabbricazione di un mezzo deve essere antecedente o uguale all'anno corrente \\
    \hline
    V6 & L'anno di assunzione di un centralinista deve essere antecedente o uguale all'anno corrente \\
    \hline
    V7 & La data di costituzione di una squadra deve essere antecedente o uguale alla data corrente \\
    \hline
    V8 & La data dell'ultima manutenzione di un mezzo deve essere antecedente o uguale alla data corrente \\
    \hline
    V9 & La priorità di un intervento deve essere compresa in un range definito (es. 1-5) \\
    \hline
    V10 & Il tipo di manutenzione deve essere "ordinaria" o "straordinaria" \\
    \hline
    V11 & La classe di un mezzo deve appartenere a un insieme predefinito (terrestre, aereo, marino) \\
    \hline
    V12 & Lo stipendio di un dipendente deve essere un valore positivo \\
    \hline
    V13 & La spesa per una manutenzione deve essere un valore positivo \\
    \hline
    V14 & Un vigile può appartenere a una sola squadra alla volta \\
    \hline
    V15 & Una squadra deve avere almeno un vigile assegnato \\
    \hline
    V16 & Ogni Caserma deve avere almeno un Dipendente. \\
    \hline
    V17 & Ogni Dipendente deve lavorare in una e una sola Caserma. \\
    \hline
    V18 & Ogni Chiamata deve essere ricevuta da uno e un solo Centralinista. \\
    \hline
    V19 & Ogni Intervento deve essere assegnato ad una e una sola Squadra. \\
    \hline
    V20 & Ogni Squadra deve essere supervisionata da uno e un solo Personale Amministrativo. \\
    \hline
    V21 & Una Manutenzione deve riguardare uno e un solo Mezzo. \\
    \hline
    V22 & Ogni Mezzo impiegato in un Intervento deve essere disponibile prima dell'inizio dell'Intervento stesso. \\
    \hline
    V23 & Ogni Dipendente deve essere esattamente uno tra Vigile, Centralinista o Personale Amministrativo (Generalizzazione Totale e Disgiunta). \\
    \hline
    V24 & Un Mezzo non può essere impiegato contemporaneamente in due Interventi distinti. \\
        \hline
\end{longtable}