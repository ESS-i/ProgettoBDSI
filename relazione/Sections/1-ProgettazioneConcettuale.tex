\section{Progettazione Concettuale}
\subsection{Analisi della richiesta}

\begin{table}[h!]
    \centering
    \begin{tabular}{| p{\textwidth} |}
        \hline
        \multicolumn{1}{|c|}{\textbf{Caserme dei Vigili del Fuoco}} \\
        \hline
        Il Corpo Nazionale dei Vigili del fuoco ha richiesto la progettazione di una base di dati relativa alla gestione operativa e amministrativa delle varie caserme dislocate nelle varie province e città italiane.
        Si vuole che ogni \textcolor{olive}{\textbf{caserma}} sia identificata da un \textit{codice numerico univoco} e siano specificati il suo \textit{indirizzo} e un \textit{numero telefonico} tramite il quale essa è raggiungibile.
        In primo luogo si rende necessario rappresentare i \textcolor{olive}{\textbf{dipendenti}} rilevanti per il contesto: \textcolor{olive}{\textbf{vigili}}, \textcolor{olive}{\textbf{centralinisti}} e \textcolor{olive}{\textbf{personale amministrativo}}.
        Per ognuno di essi, si desidera memorizzare \textit{nome}, \textit{cognome}, stipendio percepito e caserma di riferimento.
        Per quanto concerne i vigili, essi \underline{apparterranno} ad un determinato reparto e possiederanno uno specifico \textit{grado}.
        I centralinisti oltre ai dati comuni, devono avere registrati \textit{l'anno di assunzione} e \textit{l'orario del turno lavorativo}, in quanto considerati lavoratori part-time.
        Il loro compito primario è \underline{ricevere} le \textcolor{olive}{\textbf{chiamate}} di emergenza.
        Ogni chiamata è identificata in modo univoco da un \textit{codice} e deve includere \textit{orario di inizio e fine}, il \textit{numero telefonico del chiamante} e la \textit{descrizione} sintetica dell'emergenza.
        Ogni chiamata può risolversi in due modi differenti: \underline{generare} un singolo \textcolor{olive}{\textbf{intervento}} o meno.
        Di ogni intervento (identificato anch’esso tramite un \textit{codice} univoco) si vuole tener traccia del suo \textit{tipo} e della sua \textit{priorità} nonché dell’\textit{orario di inizio} (a partire dalla fine della chiamata) e di \textit{fine}, con eventuale \textit{esito} dell’operazione.
        A supporto dell’operazione, possono essere \underline{impiegati} uno o più \textcolor{olive}{\textbf{mezzi}}, descritti per \textit{classe} (ad esempio: terrestre, aereo e marino), \textit{modello}, \textit{anno di fabbricazione} e \textit{data dell’ultima \textcolor{olive}{\textbf{manutenzione}} effettuata}.
        Queste ultime, \underline{richieste} dal personale amministrativo e svolte da \textcolor{olive}{\textbf{officine}} convenzionate in una determinata \textit{data}, riguardano uno specifico mezzo, possono essere richieste in via \textit{ordinaria} (dopo un certo tempo dall’ultima revisione) o \textit{straordinaria} (in caso di guasti) e comportano una determinata \textit{spesa}.
        Ci interessa anche avere una breve \textit{descrizione} dell’intervento.
        Ogni vigile \underline{appartiene} ad una \textcolor{olive}{\textbf{squadra}}, la quale \underline{afferisce} ad un particolare amministrativo che la supervisiona.
        Essa inoltre è caratterizzata da un \textit{codice} che la identifica, una \textit{denominazione operativa}, \textit{l’area di competenza} e \textit{la data di costituzione del gruppo}, per determinarne l’anzianità. \\
        \hline
    \end{tabular}
\end{table}


Dopo aver letto il testo comprensivo delle richieste, procediamo con una prima strutturazione dei requisiti in
gruppi di frasi omogenee, per rendere la progettazione più semplice:
\subsection{Strutturazione dei requisiti}
\begin{table}[h!]
    \centering
    \begin{tabular}{| p{\textwidth} |}
        \hline
        \multicolumn{1}{|c|}{\textbf{Frasi di carattere generale}} \\
        \hline
        Il Corpo Nazionale dei Vigili del fuoco ha richiesto la progettazione di una base di dati relativa alla gestione operativa e amministrativa delle varie caserme dislocate nelle varie province e città italiane. \\
        \hline
    \end{tabular}
\end{table}

\begin{table}[h!]
    \centering
    \begin{tabular}{| p{\textwidth} |}
        \hline
        \multicolumn{1}{|c|}{\textbf{Frasi relative a Caserma}} \\
        \hline
        Si vuole che ogni caserma sia identificata da un codice numerico univoco e siano specificati il suo indirizzo e un numero telefonico tramite il quale essa è raggiungibile. \\
        \hline
    \end{tabular}
\end{table}

\begin{table}[h!]
    \centering
    \begin{tabular}{| p{\textwidth} |}
        \hline
        \multicolumn{1}{|c|}{\textbf{Frasi relative a Dipendenti}} \\
        \hline
        In primo luogo si rende necessario rappresentare i dipendenti rilevanti per il contesto: vigili, centralinisti e personale amministrativo.
        Per ognuno di essi, si desidera memorizzare nome, cognome, stipendio percepito e caserma di riferimento. \\
        \hline
    \end{tabular}
\end{table}

\begin{table}[h!]
    \centering
    \begin{tabular}{| p{\textwidth} |}
        \hline
        \multicolumn{1}{|c|}{\textbf{Frasi relative a Vigili}} \\
        \hline
        Per quanto concerne i vigili, essi apparterranno ad un determinato reparto e possiederanno uno specifico grado. \\
        \hline
    \end{tabular}
\end{table}

\begin{table}[h!]
    \centering
    \begin{tabular}{| p{\textwidth} |}
        \hline
        \multicolumn{1}{|c|}{\textbf{Frasi relative a Centralinisti}} \\
        \hline
        I centralinisti oltre ai dati comuni, devono avere registrati l'anno di assunzione e l'orario del turno lavorativo, in quanto considerati lavoratori part-time.
        Il loro compito primario è ricevere le chiamate di emergenza. \\
        \hline
    \end{tabular}
\end{table}

\begin{table}[h!]
    \centering
    \begin{tabular}{| p{\textwidth} |}
        \hline
        \multicolumn{1}{|c|}{\textbf{Frasi relative a Chiamata}} \\
        \hline
        Ogni chiamata è identificata in modo univoco da un codice e deve includere orario di inizio e fine, il numero telefonico del chiamante e la descrizione sintetica dell'emergenza.
        Ogni chiamata può risolversi in due modi differenti: generare un singolo intervento o meno. \\
        \hline
    \end{tabular}
\end{table}

\begin{table}[h!]
    \centering
    \begin{tabular}{| p{\textwidth} |}
        \hline
        \multicolumn{1}{|c|}{\textbf{Frasi relative a Intervento}} \\
        \hline
        Di ogni intervento (identificato anch’esso tramite un codice univoco) si vuole tener traccia del suo tipo e della sua priorità nonché dell’orario di inizio (a partire dalla fine della chiamata) e di fine, con eventuale esito dell’operazione.
        Ci interessa anche avere una breve descrizione dell’intervento. \\
        \hline
    \end{tabular}
\end{table}

\begin{table}[h!]
    \centering
    \begin{tabular}{| p{\textwidth} |}
        \hline
        \multicolumn{1}{|c|}{\textbf{Frasi relative a Mezzi}} \\
        \hline
        A supporto dell’operazione, possono essere impiegati uno o più mezzi, descritti per classe (ad esempio: terrestre, aereo e marino), modello, anno di fabbricazione e data dell’ultima manutenzione effettuata. \\
        \hline
    \end{tabular}
\end{table}

\begin{table}[h!]
    \centering
    \begin{tabular}{| p{\textwidth} |}
        \hline
        \multicolumn{1}{|c|}{\textbf{Frasi relative a Manutenzione e Officine}} \\
        \hline
        Queste ultime, richieste dal personale amministrativo e svolte da officine convenzionate in una determinata data, riguardano uno specifico mezzo, possono essere richieste in via ordinaria (dopo un certo tempo dall’ultima revisione) o straordinaria (in caso di guasti) e comportano una determinata spesa. \\
        \hline
    \end{tabular}
\end{table}

\begin{table}[h!]
    \centering
    \begin{tabular}{| p{\textwidth} |}
        \hline
        \multicolumn{1}{|c|}{\textbf{Frasi relative a Squadra}} \\
        \hline
        Ogni vigile appartiene ad una squadra, la quale afferisce ad un particolare amministrativo che la supervisiona.
        Essa inoltre è caratterizzata da un codice che la identifica, una denominazione operativa, l’area di competenza e la data di costituzione del gruppo, per determinarne l’anzianità. \\
        \hline
    \end{tabular}
\end{table}


\subsection{Raffinazione dei concetti, costruzione dello schema concettuale}
Iniziamo con la progettazione concettuale dei vari dati, seguendo un approccio \textit{top-down}
e partendo da semplici pattern, aggiungendo mano mano quanto necessario\dots

In primo luogo siamo partiti da 3 differenti micro-schemi di questo tipo:
\begin{figure}[H]
    \centering
    \includegraphics[width=0.7\textwidth]{Schemas/Schema1.pdf}
    \caption*{Schemi 1, 2, 3}
\end{figure}

in cui abbiamo preso in considerazione un sottoinsieme delle varie entità individuando le relazioni fondamentali che le legano.
In particolare si noti come la base dovrebbe occuparsi della gestione di 3 ambiti diversi: gestione chiamate, interventi e richiesta di manutenzione.

Osserviamo come i \textbf{vigili}, i \textbf{centralinisti} e il \textbf{personale amministrativo} possano essere accorpati all’interno della generalizzazione (dalle specifiche totale e sovrapposta) \textbf{Dipendente} che riunisce le caratteristiche (attributi) comuni, come il \textit{nome}, il \textit{cognome}, lo \textit{stipendio} percepito e la \textbf{\textit{caserma}} in cui lavorano.
Risulta il seguente:
\begin{figure}[H]
    \centering
    \includegraphics[width=1.0\textwidth]{Schemas/Schema4.pdf}
    \caption*{Schema 4}
\end{figure}

A questo punto ci concentriamo sulla parte relativa a tutti gli aspetti di un intervento, notando in primo luogo come una chiamata possa generare o meno un intervento (relazione che quindi lega queste entità) e come quest’ultimo venga gestito da una squadra, ossia da un insieme di vigili che modelliamo come entità ulteriore.

Infine, la parte amministrativa. Il personale amministrativo come già detto ha due ruoli:
\begin{itemize}
    \item Richiede le manutenzioni che verranno \textit{effettuate} da un'\textbf{officina} (che le realizza) su dei \textbf{mezzi} (modellati come entità), \textit{impiegati} nell’ambito degli interventi
    \item È responsabile della squadra, a cui essa \textit{afferisce}
\end{itemize}

A seguito di integrazione e della naturale modellazione di queste ultime tre relazioni e dell’aggiunta dell’entità \textbf{officina} che \underline{effettua} la manutenzione (nonché convenzionata con la caserma) otteniamo il seguente:

\begin{figure}[H]
    \centering
    \includegraphics[width=1.0\textwidth]{Schemas/Schema5.pdf}
    \caption*{Schema 5}
\end{figure}

Una volta terminato lo scheletro di base, aggiungiamo gli attributi alle varie entità.

