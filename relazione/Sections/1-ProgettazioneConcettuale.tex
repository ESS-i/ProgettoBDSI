\section{Progettazione Concettuale}
\subsection{Analisi dei requisiti e raggruppamento in insiemi omogenei}
\definecolor{lightgrey}{cmyk}{0.01,0,0,0.33}

\begin{table}[h!]
    \centering
    \begin{tabular}{| p{\textwidth} |}
        \hline
        \multicolumn{1}{|c|}{\textbf{Specifiche}} \\
        \hline
        Il Corpo Nazionale dei Vigili del fuoco ha richiesto la progettazione di una base di dati relativa alla gestione operativa e amministrativa delle varie caserme dislocate nelle varie province e città italiane.
        Si vuole che ogni \textcolor{olive}{\textbf{caserma}} sia identificata da un \textit{codice numerico univoco} e siano specificati il suo \textit{indirizzo} e un \textit{numero telefonico} tramite il quale essa è raggiungibile.
        In primo luogo si rende necessario rappresentare i \textcolor{olive}{\textbf{dipendenti}} rilevanti per il contesto: \textcolor{olive}{\textbf{vigili}}, \textcolor{olive}{\textbf{centralinisti}} e \textcolor{olive}{\textbf{personale amministrativo}}.
        Per ognuno di essi, si desidera memorizzare \textit{nome}, \textit{cognome}, \textit{stipendio} percepito e \textit{caserma di riferimento}.
        Per quanto concerne i vigili, ognuno di essi possiederà uno specifico \textit{grado}; i centralinisti, oltre ai dati comuni, devono avere registrati \textit{l'anno di assunzione} e \textit{l'orario del turno lavorativo}, in quanto considerati lavoratori part-time.
        Il loro compito primario è \underline{ricevere} le \textcolor{olive}{\textbf{chiamate}} di emergenza.
        Ogni chiamata è identificata in modo univoco da un \textit{codice} e deve includere \textit{orario di inizio e fine}, il \textit{numero telefonico del chiamante} e la \textit{descrizione} sintetica dell'emergenza.
        Ogni chiamata può risolversi in due modi differenti: \underline{generare} un singolo \textcolor{olive}{\textbf{intervento}} o meno, che sarà \textit{assegnato} ad almeno una \textcolor{olive}{\textbf{squadra}}.
        Di ogni intervento si vuole tener traccia del suo \textit{tipo} e della sua \textit{priorità} nonché dell’\textit{orario di inizio} (a partire dalla fine della chiamata) e di \textit{fine}, con eventuale \textit{esito} dell’operazione.
        A supporto dell’operazione, possono essere \underline{impiegati}, se a norma (revisionati entro 1 anno) uno o più \textcolor{olive}{\textbf{mezzi}}, descritti per \textit{classe} (ad esempio: terrestre, aereo e marino), \textit{modello}, \textit{anno di fabbricazione} e \textit{data dell’ultima \textcolor{olive}{\textbf{manutenzione}}} effettuata.
        Queste ultime, \underline{richieste} in una certa \textit{data} dal personale amministrativo (che si caratterizza per \textit{livello funzionale}, \textit{ruolo} e può avere o meno l'\textit{abilitazione alla firma} necessaria per la richiesta di manutenzioni) e \underline{realizzate} da \textcolor{olive}{\textbf{officine}} \underline{convenzionate} in una determinata \textit{data}, riguardano uno specifico mezzo, possono essere richieste in via \textit{ordinaria} (dopo un certo tempo dall’ultima revisione) o \textit{straordinaria} (in caso di guasti) e comportano una determinata \textit{spesa}.
        Ci interessa anche avere una breve \textit{descrizione} dell’intervento.
        Ogni vigile può \underline{entrare a far parte} di una squadra (gruppo di almeno 3 vigili), la quale \underline{afferisce} ad un particolare amministrativo che la supervisiona.
        Essa inoltre è caratterizzata da un \textit{codice} che la identifica, una \textit{denominazione operativa}, \textit{l’area di competenza} e \textit{la data di costituzione del gruppo}, per determinarne l’anzianità.\\
        \hline
    \end{tabular}
\end{table}


Dopo aver letto il testo comprensivo delle richieste e aver \textcolor{olive}{evidenziato} le varie entità coinvolte, \underline{sottolineato le relazioni} che le coinvolgono e \textit{osservato gli attributi che possiedono}, procediamo con una prima strutturazione dei requisiti in
gruppi di frasi omogenee, per rendere la progettazione più semplice:
\begin{table}[H]
    \centering
    \begin{tabular}{| p{\textwidth} |}
        \hline
        \multicolumn{1}{|c|}{\textbf{Frasi di carattere generale}} \\
        \hline
        Il Corpo Nazionale dei Vigili del fuoco ha richiesto la progettazione di una base di dati relativa alla gestione operativa e amministrativa delle varie caserme dislocate nelle varie province e città italiane. \\
        \hline
    \end{tabular}
\end{table}

\begin{table}[h!]
    \centering
    \begin{tabular}{| p{\textwidth} |}
        \hline
        \multicolumn{1}{|c|}{\textbf{Frasi relative alla caserma}} \\
        \hline
        Si vuole che ogni caserma sia identificata da un codice numerico univoco e siano specificati il suo indirizzo e un numero telefonico tramite il quale essa è raggiungibile. \\
        \hline
    \end{tabular}
\end{table}

\begin{table}[h!]
    \centering
    \begin{tabular}{| p{\textwidth} |}
        \hline
        \multicolumn{1}{|c|}{\textbf{Frasi relative ai dipendenti}} \\
        \hline
        In primo luogo si rende necessario rappresentare i dipendenti rilevanti per il contesto: vigili, centralinisti e personale amministrativo. \\
        Per ognuno di essi, si desidera memorizzare nome, cognome, stipendio percepito e caserma di riferimento. \\
        \hline
    \end{tabular}
\end{table}

\begin{table}[h!]
    \centering
    \begin{tabular}{| p{\textwidth} |}
        \hline
        \multicolumn{1}{|c|}{\textbf{Frasi relative ai vigili}} \\
        \hline
        Per quanto concerne i vigili, ognuno di essi possiederà uno specifico grado. \\
        Ogni vigile può entrare a far parte di una squadra. \\
        \hline
    \end{tabular}
\end{table}

\begin{table}[h!]
    \centering
    \begin{tabular}{| p{\textwidth} |}
        \hline
        \multicolumn{1}{|c|}{\textbf{Frasi relative ai centralinisti}} \\
        \hline
        I centralinisti, oltre ai dati comuni, devono avere registrati l'anno di assunzione e l'orario del turno lavorativo, in quanto considerati lavoratori part-time. \\
        Il loro compito primario è ricevere le chiamate di emergenza. \\
        \hline
    \end{tabular}
\end{table}

\begin{table}[h!]
    \centering
    \begin{tabular}{| p{\textwidth} |}
        \hline
        \multicolumn{1}{|c|}{\textbf{Frasi relative alle chiamate}} \\
        \hline
        Ogni chiamata è identificata in modo univoco da un codice e deve includere orario di inizio e fine, il numero telefonico del chiamante e la descrizione sintetica dell'emergenza. \\
        Ogni chiamata può risolversi in due modi differenti: generare un singolo intervento o meno. \\
        \hline
    \end{tabular}
\end{table}

\begin{table}[h!]
    \centering
    \begin{tabular}{| p{\textwidth} |}
        \hline
        \multicolumn{1}{|c|}{\textbf{Frasi relative agli interventi}} \\
        \hline
        L'intervento sarà assegnato ad almeno una squadra. \\
        Di ogni intervento si vuole tener traccia del suo tipo e della sua priorità nonché dell’orario di inizio (a partire dalla fine della chiamata) e di fine, con eventuale esito dell’operazione. \\
        Ci interessa anche avere una breve descrizione dell’intervento. \\
        A supporto dell’operazione, possono essere impiegati, se a norma (revisionati entro 1 anno) uno o più mezzi. \\
        \hline
    \end{tabular}
\end{table}

\begin{table}[h!]
    \centering
    \begin{tabular}{| p{\textwidth} |}
        \hline
        \multicolumn{1}{|c|}{\textbf{Frasi relative ai mezzi}} \\
        \hline
        I mezzi sono descritti per classe (ad esempio: terrestre, aereo e marino), modello, anno di fabbricazione e data dell’ultima manutenzione effettuata. \\
        \hline
    \end{tabular}
\end{table}

\begin{table}[h!]
    \centering
    \begin{tabular}{| p{\textwidth} |}
        \hline
        \multicolumn{1}{|c|}{\textbf{Frasi relative alla manutenzione, alle officine e all'amministrativo}} \\
        \hline
        Le manutenzioni sono richieste in una certa data dal personale amministrativo (che si caratterizza per livello funzionale, ruolo e può avere o meno l'abilitazione alla firma necessaria per la richiesta di manutenzioni). \\
        Le manutenzioni sono realizzate da officine convenzionate in una determinata data. \\
        Le manutenzioni riguardano uno specifico mezzo, possono essere richieste in via ordinaria (dopo un certo tempo dall’ultima revisione) o straordinaria (in caso di guasti) e comportano una determinata spesa. \\
        \hline
    \end{tabular}
\end{table}

\begin{table}[H]
    \centering
    \begin{tabular}{| p{\textwidth} |}
        \hline
        \multicolumn{1}{|c|}{\textbf{Frasi relative alla squadra}} \\
        \hline
        Ogni vigile può entrare a far parte di una squadra (gruppo di almeno 3 vigili). \\
        La squadra afferisce ad un particolare amministrativo che la supervisiona. \\
        Essa inoltre è caratterizzata da un codice che la identifica, una denominazione operativa, l’area di competenza e la data di costituzione del gruppo, per determinarne l’anzianità. \\
        \hline
    \end{tabular}
\end{table}

\subsection{Raffinazione dei concetti, costruzione dello schema concettuale}

Iniziamo con la progettazione concettuale dei vari dati, seguendo un approccio \textit{top-down}
e partendo da semplici pattern, aggiungendo mano mano quanto necessario\dots

In primo luogo siamo partiti da 3 differenti micro-schemi di questo tipo:
\begin{figure}[H]
    \centering
    \includegraphics[width=0.8\textwidth]{Schemas/Schema1.pdf}
    \caption*{Schemi 1, 2, 3}
\end{figure}

in cui abbiamo preso in considerazione un sottoinsieme delle varie entità individuando le relazioni fondamentali che le legano.
In particolare si noti come la nostra base di dati dovrebbe occuparsi della gestione di 3 ambiti diversi: gestione chiamate, interventi e richiesta di manutenzione.

Osserviamo come i \textbf{vigili}, i \textbf{centralinisti} e il \textbf{personale amministrativo} possano essere accorpati all’interno della generalizzazione (dalle specifiche totale ed esclusiva) \textbf{Dipendente} che riunisce le caratteristiche (attributi) comuni, come il \textit{nome}, il \textit{cognome}, lo \textit{stipendio} percepito e la \textbf{\textit{caserma}} in cui lavorano.
Risulta il seguente:
\begin{figure}[H]
    \centering
    \includegraphics[width=0.8\textwidth]{Schemas/Schema4.pdf}
    \caption*{Schema 4}
\end{figure}

A questo punto ci concentriamo sulla parte relativa a tutti gli aspetti di un intervento, notando in primo luogo come una chiamata possa generare o meno un intervento (relazione che quindi lega queste entità) e come quest’ultimo venga gestito da una squadra, ossia da un insieme di vigili che modelliamo come entità ulteriore.

Infine, la parte amministrativa. Il personale amministrativo come già detto ha due ruoli:
\begin{itemize}
    \item \underline{Richiede} le manutenzioni che verranno \underline{effettuate} da un'\textbf{officina} (che le realizza) su dei \textbf{mezzi} (modellati come entità), \underline{impiegati} nell’ambito degli interventi
    \item È responsabile della squadra, a cui essa \underline{afferisce}
\end{itemize}

A seguito di integrazione e della naturale modellazione di queste ultime tre relazioni e dell’aggiunta dell’entità \textbf{officina} che \underline{effettua} la manutenzione (nonché convenzionata con la caserma) otteniamo il seguente:

\begin{figure}[H]
    \centering
    \includegraphics[width=1.0\textwidth]{Schemas/Schema5.pdf}
    \caption*{Schema 5}
\end{figure}

Una volta terminato lo scheletro di base, aggiungiamo gli attributi alle varie entità, lo schema risultante è il seguente:

\begin{figure}[H]
    \centering
    \includegraphics[width=1.0\textwidth]{Schemas/Schema6.pdf}
    \caption*{Schema 6}
\end{figure}

Alcune osservazioni da fare:
\begin{itemize}
    \item Abbiamo preferito aggiungere un \textit{CodDip} in quanto abbiamo bisogno di un modo per identificare univocamente un dipendente e le specifiche non fornivano alcuna richiesta esplicita. 
    \item Abbiamo preferito aggiungere alcuni attributi relativi all'officina in quanto di potenziale interesse (\textit{IndOff, TelOff, NomeOff})
    \item Abbiamo aggiunto un \textit{CodInt} per identificare univocamente un intervento. In questo caso una eventuale identificazione esterna con la Chiamata sarebbe risultata problematica.
    \item Sebbene in prima battuta ci sembrasse ragionevole un'identificazione esterna della manutenzione tramite il mezzo, essa risulterebbe problematica nella gestione dello storico delle manutenzioni.\\ Aggiungiamo quindi un attributo \textit{CodManut} per facilitare il tutto. 
\end{itemize}

A questo punto andiamo inoltre a determinare le cardinalità delle associazioni ed apportiamo le modifiche necessarie allo schema:

\begin{figure}[H]
    \centering
    \hspace*{-3.5cm}
    \includegraphics[width=1.5\textwidth]{Schemas/Schema7.pdf}
    \caption*{Schema 7}
\end{figure}

\pagebreak
\subsection{Entità coinvolte}
Passiamo ora in rassegna quelle che sono le entità coinvolte, già \textcolor{olive}{\textbf{evidenziate}} nelle specifiche.

\begin{table}[H]
    \centering
    % Uso tabularx per adattare la tabella alla larghezza del testo
    \begin{tabularx}{\textwidth}{|l|X|X|}
        \hline
        \rowcolor{lightgrey}
        \multicolumn{1}{|c|}{\textbf{Entità}} & \multicolumn{1}{c|}{\textbf{Descrizione}} & \multicolumn{1}{c|}{\textbf{Attributi}} \\
        \hline
        \textbf{Caserma} & Struttura operativa dei vigili del fuoco dislocata sul territorio & \textbf{CodCaserma}, IndCaserma, NumTelCaserma \\
        \hline
        \textbf{Dipendente} & Generalizzazione del personale che lavora nella caserma & \textbf{CodDip}, NomeDip, CognomeDip, Stipendio \\
        \hline
        \textbf{Vigile} & Personale operativo che interviene nelle emergenze & GradoVigile \\
        \hline
        \textbf{Centralinista} & Personale che riceve le chiamate di emergenza & AnnoAssunzione, OrarioTurno \\
        \hline
        \textbf{Personale Amministrativo} & Personale che gestisce gli aspetti burocratici e amministrativi, come supervisione di interventi e ordine di revisioni su mezzi & MansioneAmm, LivFunzionale, AbilitFirma  \\
        \hline
        \textbf{Chiamata} & Potenziale richiesta di intervento ricevuta dalla centrale operativa & \textbf{CodChiam}, OraInizioChiam, OraFineChiam, NumChiamante, DescrChiam \\
        \hline
        \textbf{Intervento} & Operazione di soccorso effettuata a seguito di una chiamata valida & \textbf{CodInt}, TipoInt, PrioritàInt, OraInizioInt, OraFineInt, EsitoInt \\
        \hline
        \textbf{Squadra} & Gruppo di vigili che opera congiuntamente negli interventi & \textbf{CodSquad}, DenomOper, AreaComp, DataCostit \\
        \hline
        \textbf{Mezzo} & Veicolo o attrezzatura utilizzata negli interventi & \textbf{CodMezzo}, ClasseMezzo, ModelloMezzo, AnnoFabbMezzo, DataUltimaManutMezzo \\
        \hline
        \textbf{Manutenzione} & Attività di controllo e riparazione effettuata sui mezzi & \textbf{CodManut}, DataSvolgManut, TipoManut (ordinaria/straordinaria), SpesaManut, DescrManut \\
        \hline
        \textbf{Officina} & Struttura convenzionata che effettua le manutenzioni & \textbf{NomeOff, IndOff, TelOff} \\
        \hline
    \end{tabularx}
\end{table}

In \textbf{grassetto} abbiamo riportato gli attributi identificanti le varie entità, mentre l'unica identificazione esterna come già detto è stata rimossa.

\subsection{Relazioni e regole di vincolo}
Di seguito riportiamo le associazioni individuate tra le entità, con le relative cardinalità derivate dall'analisi delle specifiche e i vari vincoli di integrità individuati.

\begin{table}[h!]
    \centering
    \begin{tabularx}{\textwidth}{|p{3.5cm}|p{3.5cm}|X|}
        \hline
        \rowcolor{lightgrey}
        \multicolumn{1}{|c|}{\textbf{Relazione}} & \multicolumn{1}{c|}{\textbf{Entità e Cardinalità}} & \multicolumn{1}{c|}{\textbf{Descrizione breve}} \\
        \hline
        \textbf{APPARTENENZA} & \textbf{Vigile} (1,1) \newline \textbf{Squadra} (3,N) & Un vigile appartiene ad una singola squadra, una squadra deve essere fatta da almeno 3 vigili  \\
        \hline
        \textbf{ASSEGNATO A} & \textbf{Squadra} (0,N) \newline \textbf{Intervento} (1,N) & Una squadra può o meno essere assegnato ad un determinato intervento, ma ogni intervento deve essere assegnato almeno ad una squadra \\
        \hline
        \textbf{IMPIEGATO IN} & \textbf{Mezzo} (0,N) \newline \textbf{Intervento} (1,N)  & Un mezzo può essere usato in un intervento e in un intervento sarà usato almeno un mezzo \\
        \hline
        \textbf{EFFETTUATA SU} & \textbf{Manutenzione} (1,1) \newline \textbf{Mezzo} (0,N) & La manutenzione riguarda un singolo mezzo e sul mezzo nel tempo possono essere effettuate tante manutenzioni \\
        \hline
        \textbf{REALIZZATA DA} & \textbf{Manutenzione} (1,1) \newline \textbf{Officina} (0,N) & Una manutenzione è realizzata da una sola officina, un'officina può effettuare più manutenzioni o nessuna \\
        \hline
        \textbf{RICHIEDE} & \textbf{Manutenzione} (1,1) \newline \textbf{Amministrativo} (0,N) & Un amministrativo può eventualmente (se autorizzato) richiedere una o più manutenzioni. Una manutenzione può essere richiesta da un solo amministrativo \\
        \hline
        \textbf{AFFERENZA} & \textbf{Squadra} (1,1) \newline \textbf{Amministrativo} (0,N) & Una squadra deve per forza avere un solo amministrativo che ne è responsabile. Un amministrativo può gestire anche più squadre \\
        \hline
        \textbf{RICEVE} & \textbf{Centralinista} (0,N) \newline \textbf{Chiamata} (1,1) & Ogni chiamata è ricevuta da uno e un solo centralinista, un centralinista può ricevere zero o molte chiamate \\
        \hline
        \textbf{GENERA} & \textbf{Chiamata} (0,1) \newline \textbf{Intervento} (1,1) & Una chiamata può eventualmente generare un intervento. Un intervento è necessariamente generato da una singola chiamata \\
        \hline
        \textbf{LAVORA} & \textbf{Dipendente} (1,1) \newline \textbf{Caserma} (1,N) & In una caserma lavora almeno 1 dipendente, un dipendente lavora in una sola caserma. \\
        \hline
        \textbf{CONVENZIONE} & \textbf{Caserma} (1,N) \newline \textbf{Officina} (1,N) & Una caserma è convenzionata con almeno 1 officina, un officina è convenzionata con almeno una caserma \\
        \hline
    \end{tabularx}
\end{table}

\newpage

Durante la fase di progettazione sono stati individuati dei vincoli non esprimibili direttamente tramite il diagramma E-R. Tali vincoli, che dovranno essere garantiti probabilmente tramite trigger sono riportati nella seguente tabella:

\begin{longtable}{|c|p{\dimexpr\textwidth-3cm\relax}|}
    \hline
    \rowcolor{lightgrey}
    \textbf{ID} & \textbf{Descrizione del Vincolo} \\
    \hline
    \endfirsthead
    \hline
    \rowcolor{lightgrey}
    \textbf{ID} & \textbf{Descrizione del Vincolo (continua)} \\
    \hline
    \endhead
    \hline
    \endfoot

    V1 & L'orario di fine (chiamata o intervento) deve essere sempre successivo all'orario di inizio. Inoltre, \textit{OraInizioInt} deve essere successiva all'\textit{OraFineChiam} della chiamata che lo ha generato. \\
    \hline
    V2 & La \textit{DataSvolg} di una manutenzione deve essere uguale o successiva alla \textit{DataRichiestaManut}. \\
    \hline
    V3 & Le date di nascita, assunzione, costituzione squadra, fabbricazione mezzo e ultima manutenzione non possono essere successive alla data corrente. \\
    \hline
    V4 & Gli importi monetari (stipendi, \textit{SpesaManut}) e le priorità degli interventi devono assumere valori positivi e coerenti con il dominio (es. priorità 1-5). \\
    \hline
    V5 & Una squadra deve essere composta da almeno 3 vigili attivi \footnote{Nella versione finale dello script non abbiamo controllato il soddisfacimento di questo vincolo: per farlo avremmo semplicemente dovuto aggiungere un trigger per gestire il tutto}. \\
    \hline
    V6 & Solo il personale amministrativo con l'attributo \textit{AbilitFirma} impostato a \textit{true} può effettuare richieste di manutenzione. \\
    \hline
    V7 & Un mezzo può essere impiegato in un intervento solo se risulta "a norma", ovvero se la \textit{DataUltimaManut} non è antecedente di oltre un anno rispetto alla data dell'intervento. \\
    \hline
\end{longtable}
