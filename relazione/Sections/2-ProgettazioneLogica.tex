\section{Progettazione Logica}

In questa fase si effettua la riorganizzazione dello schema concettuale per derivare il modello logico. Si procede analizzando le prestazioni per valutare l'opportunità di introdurre o mantenere ridondanze, per poi passare alla ristrutturazione dello schema e alla traduzione finale verso il modello relazionale.

\subsection{Analisi delle prestazioni}
L'analisi delle prestazioni ha l'obiettivo di verificare se lo schema concettuale è in grado di supportare il carico applicativo previsto in modo efficiente. Questo studio si basa sulla stima dei volumi dei dati e sulla frequenza delle operazioni principali.

\subsubsection{Tavola dei volumi}
Si ipotizza uno scenario operativo relativo a un comando provinciale o regionale di medie dimensioni. I volumi stimati per le entità (E) e le associazioni (R) sono riportati nella tabella seguente:

\begin{table}[H]
    \centering
    \begin{tabular}{|l|c|r|}
        \hline
        \textbf{Concetto} & \textbf{Tipo} & \textbf{Volume stimato} \\
        \hline
        Caserma & E & 20 \\
        \hline
        Dipendente & E & 1.000 \\
        \hline
        Squadra & E & 150 \\
        \hline
        Mezzo & E & 200 \\
        \hline
        Chiamata & E & 50.000 (annui) \\
        \hline
        Intervento & E & 40.000 (annui) \\
        \hline
        Manutenzione & E & 2.000 (storico) \\
        \hline
        Lavorare & R & 1.000 \\
        \hline
        Appartenenza & R & 800 \\
        \hline
        Effettuata Su & R & 2.000 \\
        \hline
        Impiegato In & R & 100.000 (storico) \\
        \hline
    \end{tabular}
    \label{tab:volumi}
\end{table}

\subsubsection{Descrizione del carico applicativo}
Sulla base dei requisiti, sono state individuate alcune operazioni importanti che il sistema deve effettuare in maniera veloce. Le riportiamo di seguito:

\begin{table}[H]
    \centering
    \begin{tabularx}{\textwidth}{|c|X|c|c|}
        \hline
        \textbf{Op.} & \textbf{Descrizione} & \textbf{Tipo} & \textbf{Freq. (giorno)} \\
        \hline
        O1 & \textbf{Ricezione chiamata:} Inserimento dei dati di una nuova chiamata di emergenza da parte del centralinista. & Scrittura & 150 \\
        \hline
        O2 & \textbf{Apertura di un intervento:} Creazione di un nuovo intervento e assegnazione della squadra operativa. & Scrittura & 120 \\
        \hline
        O3 & \textbf{Verifica idoneità di un mezzo:} Verifica della data dell'ultima revisione per confermare che un mezzo sia a norma e impiegabile in sicurezza. & Lettura & 500 \\
        \hline
        O4 & \textbf{Registrazione Manutenzione:} Inserimento dei dati relativi a una manutenzione appena conclusa per un mezzo. & Scrittura & 5 \\
        \hline
    \end{tabularx}
\end{table}

\subsubsection{Analisi delle ridondanze}
Dall'analisi dello schema concettuale sono emerse alcune potenziali ridondanze, che dopo una veloce analisi valuteremo se lasciare o meno.

\paragraph{Ridondanza 1: DataUltimaManut (su Mezzo)} \mbox{} \\
L'attributo \textit{DataUltimaManut} nell'entità \textit{Mezzo} è derivabile, in quanto corrisponde al valore massimo della \textit{DataSvolg} tra tutte le istanze di \textit{Manutenzione} collegate a quel mezzo tramite la relazione \textit{Effettuata Su}.

Confrontiamo i costi delle operazioni coinvolte (O3 e O4) assumendo che il costo di una scrittura ($S$) sia doppio rispetto a una lettura ($L$), ovvero $1S = 2L$. Si assume inoltre che ogni mezzo abbia uno storico medio di 10 manutenzioni passate.

\vspace{0.3cm}
\textbf{Scenario A: Presenza della ridondanza} \\
L'attributo è memorizzato direttamente nel \textit{Mezzo}.
\begin{itemize}
    \item \textbf{Op. O3 (Lettura):} Accesso diretto all'entità Mezzo.
    \item \textbf{Op. O4 (Scrittura):} Richiede l'inserimento della manutenzione, della relazione e l'aggiornamento dell'attributo nel mezzo.
\end{itemize}

\begin{table}[H]
    \centering
    \begin{tabular}{|l|c|c|c|}
        \hline
        \textbf{Operazione} & \textbf{Accessi} & \textbf{Costo unitario} & \textbf{Costo Totale} \\
        \hline
        O3 (Verifica Mezzo) & 1 L & $1L$ & $500 \times 1L = 500L$ \\
        \hline
        O4 (Reg. Manutenzione) & 3 S & $6L$ & $5 \times 6L = 30L$ \\
        \hline
        \multicolumn{3}{|r|}{\textbf{Totale giornaliero}} & \textbf{530 L} \\
        \hline
    \end{tabular}
\end{table}

\vspace{0.3cm}
\textbf{Scenario B: Assenza della ridondanza} \\
Il dato deve essere calcolato leggendo lo storico.
\begin{itemize}
    \item \textbf{Op. O3 (Lettura):} Accesso al mezzo, alla relazione (ricordiamo che c'è una media di 10 manutenzioni per mezzo) e scansione delle manutenzioni per trovare la data massima. Accessi stimati: 1 (Mezzo) + 10 (Relazione) + 10 (Manutenzione) = 21 accessi in lettura.
    \item \textbf{Op. O4 (Scrittura):} Inserimento della manutenzione e della relazione (2 accessi in scrittura).
\end{itemize}

\begin{table}[H]
    \centering
    \begin{tabular}{|l|c|c|c|}
        \hline
        \textbf{Operazione} & \textbf{Accessi} & \textbf{Costo unitario} & \textbf{Costo Totale} \\
        \hline
        O3 (Verifica Mezzo) & 21 L & $21L$ & $500 \times 21L = 10.500L$ \\
        \hline
        O4 (Reg. Manutenzione) & 2 S & $4L$ & $5 \times 4L = 20L$ \\
        \hline
        \multicolumn{3}{|r|}{\textbf{Totale giornaliero}} & \textbf{10.520 L} \\
        \hline
    \end{tabular}
\end{table}

Il mantenimento della ridondanza garantisce un risparmio di circa 10.000 accessi giornalieri e, soprattutto, assicura tempi di risposta immediati durante le fasi critiche di emergenza. Si decide di \textcolor{green}{\textbf{mantenere l'attributo ridondante}}.

\paragraph{Altre ridondanze mantenute} \mbox{} \\
Oltre al caso analizzato, si decide di mantenere le seguenti ridondanze per motivi di efficienza e stabilità del dato:
\begin{itemize}
    \item \textbf{Stipendio (su Dipendente):} Potremmo rendere il calcolo più intelligente, calcolando eventuali bonus legati a grado e anzianità ma non essendoci specifiche al riguardo \textcolor{green}{si mantiene come attributo}.
    \item \textbf{OraInizioInt (su Intervento):} Coincide con \textit{OraFineChiam} della chiamata che lo ha generato. Non è particolarmente oneroso ottenere la fine della chiamata, pertanto \textcolor{red}{rimuoviamo l'attributo}.
\end{itemize}
\newpage
%TODO: schema aggiornato

\subsection{Ristrutturazione dello schema E-R}
In questa fase si apportano modifiche allo schema concettuale al fine di ottimizzarlo per il modello relazionale, basandosi sulle valutazioni prestazionali e sui volumi di dati stimati.

\subsubsection{Eliminazione delle generalizzazioni}
Lo schema presenta una generalizzazione \textbf{totale e esclusiva} nell'entità \textit{Dipendente}, che si specializza in \textit{Vigile}, \textit{Centralinista} e \textit{Personale Amministrativo}.
La decisione che prendiamo è quella di creare un'unica entità \textit{Dipendente} che conterrà, oltre agli attributi comuni, anche quelli specifici (\textit{GradoVigile, RepartoVigile, AnnoAssunzione, OrarioTurno, MansioneAmm, LivFunzionale, AbilitFirma}) che assumeranno valore NULL qualora il dipendente non ricopra quel ruolo specifico.
Per effettuare delle eventuali interrogazioni sfrutteremo le viste per semplicità. Le relazioni specifiche (\textit{Appartenenza, Riceve, Richiede, Afferenza}) verranno collegate direttamente all'entità \textit{Dipendente}.

\subsubsection{Partizionamento di entità, eliminazione attributi composti e multivalore}
Nello schema sono presenti degli attributi composti usati per rappresentare gli indirizzi (delle caserme e delle officine); li scorporiamo in attributi semplici. 
Abbiamo deciso di non inserire attributi multivalore nel nostro schema.
Per quanto concerne il partizionamento di associazioni, non abbiamo avuto necessità di modifiche ulteriori.
\subsubsection{Scelta degli identificatori primari}
Per ogni entità dello schema ristrutturato, definiamo l'identificatore principale che verrà utilizzato nel modello relazionale.     

\begin{table}[H]
    \centering
    \begin{tabularx}{\textwidth}{|l|l|X|}
        \hline
        \rowcolor{lightgrey}
        \textbf{Entità} & \textbf{Identificatore scelto} & \textbf{Motivazione} \\
        \hline
        Caserma & \textit{CodCaserma} & Identificatore numerico univoco già presente nelle specifiche. \\
        \hline
        Dipendente & \textit{CodDip} & Introdotto per gestire univocamente il personale \\
        \hline
        Chiamata & \textit{CodChiam} & Identificatore univoco della chiamata, già presente nelle specifiche \\
        \hline
        Intervento & \textit{CodInt} & Introdotto per identificare univocamente l'intervento \\
        \hline
        Squadra & \textit{CodSquad} & Identificatore univoco della squadra, già presente nelle specifiche. \\
        \hline
        Mezzo & \textit{CodMezzo} & Identificatore univoco del mezzo. \\
        \hline
        Manutenzione & \textit{CodManut} & Lo avevamo già preferito rispetto all'identificazione esterna per semplicità. \\
        \hline
        Officina & \textit{CodOff} & \textbf{Nuovo identificatore.} Nello schema concettuale l'officina era identificata da attributi anagrafici (NomeOff, IndOff). Si introduce un ID numerico per efficienza nelle chiavi esterne. \\
        \hline
    \end{tabularx}
\end{table}
\pagebreak
\subsection{Schema ristrutturato}
A questo punto otteniamo il seguente schema ristrutturato:

\begin{figure}[H]
    \centering
    \hspace*{-4cm}
    \includegraphics[width=1.5\textwidth]{Schemas/Schema8.pdf}
    \caption*{Schema 8}
\end{figure}

\subsection{Verso il Modello Relazionale}
\subsubsection{Schema logico}
\textbf{Caserma}(\underline{CodCaserma}, NumTelCaserma, ViaCaserma, CivicoCaserma, CapCaserma, CittàCaserma)\\
\textbf{Convenzione}(\underline{CodCaserma}, \underline{CodOff})\\
\textbf{Dipendente}(\underline{CodDip}, NomeDip, CognomeDip, Stipendio, *GradoVigile, *AnnoAssunzione, *OrarioTurno, *MansioneAmm, *LivFunzionale, *AbilitFirma, *CodSquad, *DataEntrata, *DataUscita, CodCaserma)\\
\textbf{Squadra}(\underline{CodSquad}, DenomOper, AreaComp, DataCostit, CodCaserma)\\
\textbf{AssegnatoA}(\underline{CodSquad}, \underline{CodInt})\\
\textbf{Intervento}(\underline{CodInt}, TipoInt, PrioritàInt, OraFineInt, EsitoInt, CodChiam)\\
\textbf{ImpiegatoIn}(\underline{CodInt,CodMezzo})\\
\textbf{Mezzo}(\underline{CodMezzo}, ClasseMezzo, ModelloMezz, AnnoFabbMezzo, DataUltManut)\\
\textbf{Chiamata}(\underline{CodChiam}, OrarioInizioChiam, OrarioFineChiam, DescrChiam, NumChiamante, CodDip)\\
\textbf{Manutenzione}(\underline{CodManut}, DataSvolgManut, SpesaManut, TipoManut, DescrManut, CodDip, DataRichiesta, CodOff, CodMezzo)\\
\textbf{Officina}(\underline{CodOff}, TelOff, NomeOff, CittàOff, ViaOff, CivicoOff, CapOff)

\subsubsection{Vincoli di integrità referenziale}
Dipendente.CodCaserma $\rightarrow$ Caserma.CodCaserma\\
Dipendente.CodSquad $\rightarrow$ Squadra.CodSquad\\
Squadra.CodCaserma $\rightarrow$ Caserma.CodCaserma\\
Chiamata.CodDip $\rightarrow$ Dipendente.CodDip\\
Intervento.CodChiam $\rightarrow$ Chiamata.CodChiam\\
AssegnatoA.CodSquad $\rightarrow$Squadra.CodSquad\\
AssegnatoA.CodInt $\rightarrow$Intervento.CodInt\\
ImpiegatoIn.CodInt $\rightarrow$Intervento.CodInt\\
ImpiegatoIn.CodMezzo $\rightarrow$Mezzo.CodMezzo\\
Manutenzione.CodDip $\rightarrow$ Dipendente.CodDip\\
Manutenzione.CodOff $\rightarrow$ Officina.CodOff\\
Manutenzione.CodMezzo $\rightarrow$ Mezzo.CodMezzo\\
Convenzione.CodCaserma $\rightarrow$ Caserma.CodCaserma\\
Convenzione.CodOff $\rightarrow$ Officina.CodOff\\

% Ho preferito non usare le tabelle per una maggiore visibilità, puoi cambiarle se vuoi.
